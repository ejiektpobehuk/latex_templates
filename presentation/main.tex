\documentclass[11pt]{beamer}
\usetheme{Madrid}
\usecolortheme{seagull}
\usepackage[utf8]{inputenc}
\usepackage[russian]{babel}
\usepackage{amsmath}
\usepackage{amsfonts}
\usepackage{amssymb}
\usepackage{here}
\author[Петров В.Д.]{Петров Владислав Дмитриевич}
\title[Краткое название]{Полное название презентации}
\date{\the\year} 
\begin{document}


\begin{frame}
\titlepage
\end{frame}


\begin{frame}{Обычный слайд текста}
Текст сам центрируется по высоте слайда. Центрирование по горизонтали 
\end{frame}


\begin{frame}{Список}
\begin{itemize}
	\item Элемент списка
	\begin{itemize}
		\item Элемент вложенного списка
	\end{itemize}
	\item Элемент списка
\end{itemize}
\end{frame}


\begin{frame}{Рисунок}
\begin{figure}[H]
	\includegraphics[scale=0.4]{pics/sample}
	\label{fig:sample}
\end{figure}
\center{Текст под рисунком, не подпись}
\end{frame}


\begin{frame}{Формулы, сложна}
Спектр (спектральная плотность) $\Phi(f)$ в общем случае представляет собой комплексную функцию: $$\Phi(f)=|\Phi(f)|*e^{i\psi(f)}$$
Модуль этой функции $|\Phi(f)|$ называют спектром амплитуд, а зависимость $\psi(f)$ — спектром фаз.
\end{frame}


\end{document}
