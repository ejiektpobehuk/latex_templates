\section{Вывод}
В инвертирующем масштабном усилителе устанавливаются противоположные знаки входного и выходного напряжения. Диапазон линейной характеристики усилителя для постоянного тока лежит в пределах $\pm 13.8$ В выходного напряжения и $\pm 14.3$ В входного.

Из-за небольшого коэффициента усиления ($K=1$) не удалось точно узнать границы линейного участка характеристики для переменного тока; при $U_{вх}=10.5$ В и $U_{вых}=10.02$ В начинает наблюдаться снижение коэффициента усиления.

При слишком больших и слишком маленьких сопротивленияx первого и второго резисторов коэффициент усиления начинает отличаться от заданного, т.к. в этом случае начинают влиять характеристики, не учтённые в расчётах, например, входное сопротивление ОУ.

Для масштабирующего усилителя без инверсии фазы входное и выходное напряжение одного знака.
Коэффициент усиления для последней схемы сначала возрастает (из-за наличия конденсатора), а затем убывает начиная с некоторой частоты (из-за свойств ОУ). На линейном участке характеристики $K=1=0$ дБ, при частоте 500 Гц $K=0.69=-3.2$ дБ.