%%%%%%%%%%%%%%%%%%%%%%%%%%%%%%%%%%%%%%%%%%%%%%%%%%%%%%%%%%%%%%%%%%%%%%%%%%%%%%%%
%%
%% Настройка параметров документа
%%
%%%%%%%%%%%%%%%%%%%%%%%%%%%%%%%%%%%%%%%%%%%%%%%%%%%%%%%%%%%%%%%%%%%%%%%%%%%%%%%%

% Быть посвободнее при склеивании слов
\sloppy

% Настройка листингов
\renewcommand{\lstlistingname}{Листинг}
\lstset{
	frame=single, % adds a frame around the code
	rulesepcolor=\color{gray},
	rulecolor=\color{black},
	breaklines=true,
	xleftmargin=2em,
	extendedchars={true},
	inputencoding={utf8},
	basicstyle={\ttfamily \scriptsize},
	keywordstyle={\rmfamily \bfseries},
	commentstyle={\rmfamily \itshape},
	tabsize={2},
	numbers={left},
	frame={single},
	showstringspaces={false},
}
\lstdefinestyle{java}{
	breaklines={true},
	texcl=true,
	language={Java},
}
\input{listings_cyr_hack}

% Настройка стиля оглавления
% \renewcommand{\tocchapterfont}{}

%%%%%%%%%%%%%%%%%%%%%%%%%%%%%%%%%%%%%%%%%%%%%%%%%%%%%%%%%%%%%%%%%%%%%%%%%%%%%%%%


\begin{document}	% начало документа

% Титульная страница
%\begin{titlepage}	% начало титульной страницы

	\begin{center}		% выравнивание по центру

		\large Санкт-Петербургский Политехнический Университет Петра Великого\\
		\large Институт компьютерных наук и технологий \\
		\large Кафедра компьютерных систем и программных технологий\\[6cm]
		% название института, затем отступ 6см
		
		\huge \textbf{КУРСОВОЙ ПРОЕКТ}\\[0.5cm]
		\large Название предмета\\[0.1cm]
		\large Тема работы\\[5cm]

	\end{center}


	\begin{flushright} % выравнивание по правому краю
%		\begin{minipage}{0.5\textwidth} % врезка в половину ширины текста
%			\begin{flushleft} % выровнять её содержимое по левому краю

				\large Выполнил студент группы 43501/4\\
				\large В.Д. Петров\\[0.5cm]
				
				\large Принял к.т.н., доцент\\
				\sign[4cm]\large  В.М. Ицыксон\\
				\large Оценка: \sign\\
				«\underline{\hspace{0.7cm}}» \underline{\hspace{2cm}} \the\year г.

%			\end{flushleft}
%		\end{minipage}
	\end{flushright}
	
	\vfill % заполнить всё доступное ниже пространство

	\begin{center}
	\large Санкт-Петербург\\
	\large \the\year % вывести дату
	\end{center} % закончить выравнивание по центру

\thispagestyle{empty} % не нумеровать страницу
%\end{titlepage} % конец титульной страницы
\newpage


% Содержание
% Содержание
\renewcommand\contentsname{\centerline{Содержание}}
\tableofcontents
\newpage




\section{Цель работы}


\section{Программа работы}


\section{Теоретическая информация}


\section{Ход выполнения работы}

\subsection{Список}

\begin{itemize}
\item первый элемент списка
\item второй элемент списка
\end{itemize}


\subsection{Картинка}

\begin{figure}[H]
	\begin{center}
		\includegraphics[scale=0.7]{sample}
		\caption{название картинки} 
		\label{pic:pic_name} % название для ссылок внутри кода
	\end{center}
\end{figure}


\subsection{Листинг}

\lstinputlisting[
	label=code:hello,
	caption={hell\_o.c},% для печати символ '_' требует выходной символ '\'
]{hell_o.c}
\parindent=1cm % командна \lstinputlisting сбивает параментры отступа
Текст без отступа (следует за вставкой)

Новый параграф

\noindent Новый параграф с принудительно выключенным отступом


\subsection{Частичный листинг}
% настрока частичного ввода (требуется один раз)
\makeatletter
\def\lst@PlaceNumber{\llap{\normalfont
                \lst@numberstyle{\the\lst@lineno}\kern\lst@numbersep}}
\makeatother

\lstinputlisting[
	label=code:hello_mod,
	linerange={4-5},
	caption={фрагмент hell\_o.c},
]{hell_o.c}
\parindent=1cm

\subsection{Таблица}

\begin{table}[H]
	\caption{ Название таблицы}
	\begin{center}
		\begin{tabular}{|l|l|}
			\hline
			top left & top right\\ \hline
			bot left & bot right\\ \hline
		\end{tabular}
		\label{tabular:tab_examp}
	\end{center}
\end{table}

\section{Выводы}
\LaTeX\ удобен для создания отчётов, так как сам следит за нумерацией таблиц, рисунков, листингов и отсылок к ним (так, например, здесь всегда будет указан номер рисунка "sample" не зависимо от того, какой он (1,2 или другой) - это рисунок \ref{pic:pic_name}). Не менее важно что весь документ оформлен в едином стиле, а исходные материалы подключаются к отчёту, а не хранятся в нём. Всё это позволяет легко получить качественный отчёт без дополнительных трат на его офрмление.

Исключения, пожалуй, составляют таблицы, так как их значительно сложнее создавать кодом, нежели в графическом редакторе. Но здесь никто не запрещает использовать визуальные средства создания таблиц для \LaTeX\ .
\end{document}
